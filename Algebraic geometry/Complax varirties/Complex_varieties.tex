\documentclass[a4paper]{article}
  \usepackage{amsmath}
  \usepackage{amssymb}
  \usepackage{setspace}
  \usepackage{enumitem}
  \usepackage{geometry}
  \usepackage{amsthm}
  \usepackage[all,pdf]{xy}
  \newtheorem{thm}{Theorem}
  \newtheorem{lemma}[thm]{Lemma}
  \newtheorem{cor}[thm]{Corllary}
  \newtheorem{prop}[thm]{Proposition}
  \theoremstyle{definition}
  \newtheorem{defi}[thm]{Definition}
  \theoremstyle{remark}
  \newtheorem{rmk}{Remark}


 \geometry{a4paper,left=2cm,right=2cm,top=1cm,bottom=3cm}
  \author{Xi Wang}
  \title{Complex varieties}
  \date{November 23,2016}
\begin{document}
  \maketitle
  Suppose now, our algebraically closed ground field is given a topology, making it into a topological field. The most interesting case is when k = $\mathbb{C}$. By equipping all varieties over $\mathbb{C}$ with its analytical topology(which is called the \emph{strong topology} in our following statement), we notice that the following properties hold:
  \begin{enumerate}[itemsep=0pt,parsep=0.2pt]
    \item the strong topology is stronger than the Zaraski-topology i.e.,Zaraski closed(resp. open) sets are closed(resp. open) in strong topology.
    \item all morphisms are strongly continuous.
    \item the strong topology of a locally closed (i.e.,intersection of an open and a closed set) subvariety $X \subset Z$ is the induced topology on $X$.
    \item the strong topology on $X \times Y$ is the product of strong topologies on $X$ and $Y$.\vspace{4ex}
  \end{enumerate}


  \noindent\fbox{$\bold{Recall:}$}

  \begin{thm}[Noether's normalization, geometric form]
    Suppose $X$ is an affine variety (of dim n), then there exist a surjective morphism: \[\pi: X \longrightarrow \mathbb{A}^n, \] and for every $P\in \mathbb{A}^n$, $\pi^{-1}(P)$ is a finite but nonempty set.(In the following statement, we will call such morphism a finite morphism)
       \begin{proof}
          (existence) Embed $X$ into $\mathbb{A}^n$. Its coordinate ring is noted as $A:=k[x_1,x_2,\dots,x_m]/I$. By Noether's normalization, there exists $y_1,y_2,\dots,y_n\in A$ which are algebraically independent, and, when the field contains infinite elements, are of the form of linear combination of $\overline{x}_1,\dots,\overline{x}_m$, making $A$ a finite $k[y_1,y_2,\dots,y_n]$ algebra. Then the inclusion $\pi^*:k[y_1,y_2,\dots,y_n]  \hookrightarrow A$ induce the morphism: \[\pi: X \longrightarrow \mathbb{A}^n\]

          (finite) As $\overline{x}_i$ is integral on $K:=k[y_1,y_2,\dots,y_n]$ ,so we have \[\overline{x}_i^N+f^i_N\overline{x}_i^{N-1}+\dots+f_0^i = 0, \ f^i_j \in K\] when $\{y_i\}$ is given, the solution is finite.\vspace{2ex}

          (nonempty) Let $P=(b_1,\dots,b_n)$, It suffice to show $V((y_1-b_1,\dots,y_n-b_n)) \not= A$, applying nakayama's theorem on $K\subset A$, we know it is true.
       \end{proof}
  \end{thm}

  \begin{lemma}
    [Hausdoff axiom] $X$ is a variety $\Rightarrow \Delta(X):=\{(x,x)|x\in X\}$ is closed.
  \end{lemma}
  \begin{rmk}
     It sometimes becomes the definition of the variety, in other words, like the definition of manifolds, a
     (abstract) variety $X$ is a locally affine topological space satisfying the Hausdoff axiom.
  \end{rmk}
  \begin{prop}TFAE:
    \begin{enumerate}[itemsep=0pt,parsep=0.2pt]
        \item $\Delta(X)$ is closed.
        \item for any morphism $f,g$: $Y\rightarrow X, \ \{y\in Y|f(y)=g(y)\}$ is closed.
        \item for any morphism $f$: $Y\rightarrow X$, the graph of $f$ is closed.(i.e., the image of \ $ Y\xrightarrow{(f,id)}X\times Y$ is closed)\vspace{2ex}
    \end{enumerate}
  \end{prop}

  The following definition and theorem is mentioned by HY Gao.

  \begin{defi}[Complete variety]
    $X$ is a complete variety if for any variety $Y$, $X\times Y \rightarrow Y$ is a closed map.
  \end{defi}
  \begin{thm}
    Any projective variety is complete.\vspace{2ex}
  \end{thm}

  Our work starts here. The first comparison of two topologies states that strong topology is not "too strong".
  \begin{thm} Let $X$ be a irreducible variety, and  $U$ a nonempty open subvariety. Then $U$ strongly dense in $X$.
    \begin{proof}
        Since the statement is local, we can suppose $ X $ is affine with coordinate ring $ A  \subset   k[x_1,   \dots,   x_m]  $, By Noether's theorem, we can find $K: =k [y_1,\dots,y_n]$ s.t, $A$ is a finite algebra over $K$ and there exist a finite surjective map
                     \[\pi: X \rightarrow \mathbb{A}^n.\]
         Let $ Z = X   \setminus U$ then $Z$  be Zaraski closed $i.e., Z=V(f_1,\dots,f_n)$. Choose $h\in I(Z)$, since h is integral on $K$, it satisfies an equation $h^N+f_Nh^{N-1}+\dots+f=0$, $f_i,f\in K.$When $x \in Z \Rightarrow h(x)=0 \Rightarrow f(y)=0$,therefore $\pi(Z)\subset V(f))$.
         Now fixed $x_0\in X$, we want to construct a sequence $\{y^i\}$ $\in$ $\mathbb{A}^n\setminus\pi(Z)$ strongly converge to $\pi(x_0)$. In order to do this, take $y \not \in V(f)$ and
                    \[\tilde{h}:=f((1-t)y+t\pi(x_0)), \tilde{h}(t)\not \equiv 0.\]
        Therefore $\tilde{h}(t)$ has finite zero points, so we can take $t^i \rightarrow 0$, and the corresponding $y^i:=(1-t^i)y+t^i\pi(x_0)\not \in \pi(Z)$ and $y^i \rightarrow$ $ \pi(x_0), $.
        The problem now is to lift the converge from $\mathbb{A}^n$ into $X$. Let$\pi(\pi^{-1}(x_0))=\{x_0.x^1,x^2,\dots,x^{n^\prime}\}$, there exist a $g\in A$ s.t. $g(x_0)=0$ and $g(x^i)\not = 0$ ($T_4$ property of Zaraski topology). Due to the integral dependency of $g, \exists  \ A_1,\dots,A_N \in K$ satisfy:
                    \[F(x):=g^N+A_Ng^{n-1}+\dots+A_0=0.\]
       The inclusions of
                     \[K \xrightarrow{i_1}     K[g]=k[y_1,\dots,y_n,y_{n+1}]/F \xrightarrow{i_2}      A\]
        induce a morphism chain:
        \begin{displaymath}
                \xymatrix@R=0ex@C=0ex{
                    *[l]{X \ \ }                             \ar@{^(-^>}[r]^(0.3){\pi_1}  &V(F)\subset \mathbb{A}^{n+1} \   \ar@{^(-^>}[r]^(0.7){\pi_2} & *[r]{ \ \mathbb{A}^{n}} \\
                    *[l]{(x_1,x_2,\dots,x_m) \ }    \ar@{|->}[r]^(0.3){\pi_1}        &(y_1,\dots,y_n,g(x))                       \ar@{|->}[r]^(0.7){\pi_2}      & *[r]{(y_1,\dots,y_n)}
                }
        \end{displaymath}
        
      We claim: $\pi_1,\pi_2$ is finite.
      
      Now that, $F(x_0)=F((\pi(x_0),g(x_0))=0 \Rightarrow A_0(\pi(x_0))=0$, $A_0(y^i) \rightarrow 0$. On the other hand,
                     \[A_0(y^i)=\prod_{F(y^i,t^i_k)=0}{t^i_k}.\]
      Therefore we can find roots $t^i$ of $F(y^i,t)=0$ such that  $ (y^i,t^i) \rightarrow \pi_1(x_0)$.
        \begin{spacing}{1.2}
            Similarly, since $x_1,x_2,\dots,x_n$ integral on $K[g]$, $\exists \ a_{ij}\in K[g] $ such that
                    \[ x_i^N+a_{in}x_i^{N-1}+\dots+a_{i0}=0\]
            When $\Sigma$ is a compact set $\subset V(F)$ in the strong topology, $\{a_{ij}(\Sigma)\}$ is bounded. So $\pi^{-1}(\Sigma)$ is closed and bounded strongly, thus compact strongly. Choose $\Sigma=\{(y^i,t^i)\}$. We claim that there exist a subsequence $\{z^i\} \in X$ strongly converge to $x_0$. If not, for compact set, there must be a subsequence $\{z^{i^k}\}$ converge to $x' \not =x$. then \[\pi(x')=\lim_{i^k \rightarrow \infty} \pi(z^{i^k})=\pi(x)\] therefore $x'$ must be some $x^i$, but\[t^{i^k}=\lim_{i^k \rightarrow \infty} g(z^{i^k})=0\]
\emph{Contradiction!} thus there exist a sequence in $U$ converge to $x_0$ strongly.\qed
\end{spacing}
\end{proof}
\end{thm}


\begin{cor}
If $Z \subset X$ is locally closed, the strong closure and Zaraski closure actually the same.
\end{cor}
  \noindent\underline{Proof}: $Z$ is open in $\bar{Z}^{zar}$,so strongly dense in $\bar{Z}^{zar}$. The strong topology of $\bar{Z}^{zar}$ is induced by that of $X$, therefore, $\bar{Z}^{str}=\bar{Z}^{zar}$.\qed

\begin{thm} Let $X$ be a complex variety, then $X$ in complete iff $X$ compact strongly.
\end{thm}
\begin{spacing}{1.1}
  \noindent\underline{Proof}: $(\Longrightarrow)$ Suppose $X$ is strongly compact. Assume Y is affine (moreover,$\mathbb{A}^n$), $Z\subset X\times Y$ irreducible closed set, and $p_2:X\times Y \rightarrow Y$ the projection. Since $X$ compact, $p_2$ is a proper map strongly (the inverse image of compact set is compact), with locally compactness of Y=$\mathbb{A}^n$, $P_2$ is closed in the strong topology. And as $Z$ irreducible, $p_2(Z)$ contains a Zaraski open set in $\overline{p_2(Z)}^{zar}$, by corllary1, we get $\overline{p_2(Z)}^{zar}=\overline{p_2(Z)}^{str}=p_2(Z)$.
\end{spacing}
$(\Longleftarrow)$ As we know, the projective variety over $\mathbb C$ is compact and complete. The following lemma shows us how complete varieties are related with projective varieties.
\begin{lemma}[Chow] $X$ is a complete variety over a algebraically closed field, then there exist a projective variety $Y$ and a surjective, birational map:  $$\pi:Y \rightarrow X $$
\end{lemma}
  \noindent\underline{Proof}:
 \begin{enumerate}[itemindent=41pt,leftmargin=4pt,itemsep=0pt,parsep=0.2pt,label=$\bold{\qquad STEP \ \arabic{enumi}}$]
 \item(construct $Y$) Cover X by a colletion of open affine subsets $U_i\subset\mathbb{A}^n\subset\mathbb{P}^n$. Since $k[x_1,\dots,k_n]$ is noetherian, the cover is finite. Take $\bar{U}_i$, the closure of $U_i$ in $\mathbb{P}^n$, $U^* = \bigcap_{i=1}^nU_i$ an open set.Then consider the "multidiagonal" morphism: $$U^* \xrightarrow{\Delta^n} U^*\times U^* \times \dots \times U^*\hookrightarrow \bar{U}_1\times \dots \times\bar{U}_m $$ the closure of the image in $\bar{U}_1\times \dots \times\bar{U}_m$ is denoted by $Y$. $Y$ is a projective thus complete variety. Now we want to construct the birational morphism $\pi$.
 \item(construct $\pi$)Consider following morphism: $$U^* \xrightarrow{\Delta} U^*\times U^* \xrightarrow{(i ,\ \Delta^n)} X\times Y$$ the projection restricted on the closure of image(noted as $\tilde{Y}$) give us two commutative diagrams:
        \begin{displaymath} \xymatrix{
        U^* \ar@{^{(}->}[r]  \ar[d]_{id} &\tilde{Y} \ar[d]_{p}        &U^* \ar@{^{(}->}[r]  \ar[d]_{id} &\tilde{Y} \ar[d]_{q} \\
        U^* \ar@{^{(}->}[r]              &X                           &U^* \ar@{^{(}->}[r]              &y}
       \end{displaymath}
       as $U^*$ is (Zaraski) open and dense in ${\tilde{Y}}$, $p,q$ are birational. Moreover, as X,Y are both complete, $p({\tilde{Y}}),q(\tilde{Y})$ is closed set contains $U^*$, it follows that $p,q$ are surjection. Let $\pi=p\circ q^{-1}$, if \emph{$q$ is isomorphism}, the everything is proven. By following lemma, we can prove it easily;\\
 \item(q isomorphism) first, let us look at this lemma.
       \begin{lemma} Let $S$ and $T$ be varieties, with isomorphic open subset $V$, look at the morphism:\[V\xrightarrow{\Delta}V\times V\hookrightarrow S\times T.\] If $\bar{V}$ is the closure of the image, then\[\bar{V} \cap (S\times V)=\bar{V} \cap(V\times T)=(\Delta(V)).\]
       \end{lemma}
       \noindent\underline{Proof}: It suffice to show that $\Delta(V)$ is already closed in $V\times T$ and in $S\times V$. But$\bar{V} \cap (S\times V)$ is just the graph of the inclusion morphism $V\rightarrow T$. Hence it is closed.\\
       \qquad Back to our proof, we want to analyze the projection $X\times Y$ on $X\times \bar U_i$. Look at the diagram:
       \begin{displaymath} \xymatrix{
                                                 &U^*\times U^* \dots U^* \ar@{^{(}->}[r] \ar[d]  &X\times \bar U_1 \dots \bar U_n \ar[d]_{r_i}        \\
        U^* \ar[r]^{\Delta}  \ar[ur]^{\Delta^n}  &U^*\times U^*  \ar@{^{(}->}[r]                  &X\times \bar U_i      }
       \end{displaymath}
       We claim that $r_i(\tilde Y) = \overline{\Delta(U^*)}$ because $r_i$ is a closed map. Therefore
        \begin{align*}
         r_i(\tilde Y)\cap(X\times U_i) \qquad =& \qquad r_i(\tilde Y)\cap(U_i \times \bar U_i) \\
                                         =& \qquad \{(x,x)|x \in U_i \}
       \end{align*}
       Therefore
        \[\tilde Y \cap (X\times \bar U_1 \times \dots \times U_i \times \dots \times \bar U_n)=  \tilde Y \cap (U_i\times \bar U_1 \times \dots \times \bar U_n)\]Called these set $\tilde Y_i$. From the second form it follows that $\{{\tilde Y_i\}}$ is a open covering of $\tilde Y$. From the first form of the intersection, it follows that
        \begin{align*}
        \tilde Y_i\quad&=\quad q^{-1}(Y_i)\\
        \intertext {if}
        Y_i\quad&=\quad\tilde Y \cap (\bar U_1 \times \dots \times U_i \times \dots \times \bar U_n)
        \end{align*}
        Since $q$ is surjective, this implies that $Y_i$ must be an open covering of Y, now define
        \begin{align*}
        \sigma_i&:Y_i\rightarrow \tilde Y_i \\
        \sigma_i&(u_1,\dots,u_n)=(u_i,u_1,\dots,u_m))
        \end{align*}
        Then $\sigma_i$ is an inverse of q restrict to $\tilde Y_i$:
        \begin{enumerate}
        \item $q(\sigma_i(u_1,\dots,u_n))=q(u_i,u_1,\dots,u_n)=(u_1,\dots,u_n)$
        \item by the lemma, all points $(v,u_1,\dots,u_n)$ of $\tilde Y_i$ satisfy $v=u_i$, hence \[\sigma_i(q(v,u_1,\dots,u_n))=\sigma_i(u_1,\dots,u_n)=(v,u_1,\dots,u_n)\]
        \end{enumerate}
        By gluing $\sigma_i$ together,we can see $q$ is an isomorphism, and $\pi$ is constructed.\qed\vspace{4ex}
        \end{enumerate}

       \noindent\fbox{$\bold{Complex \ and \ algebraic \ geometry}$}

       Algebraic varieties are locally defined as the common zero sets of polynomials and since polynomials over the complex numbers are holomorphic functions, algebraic varieties over C can be interpreted as analytic spaces. Similarly, regular morphisms between varieties are interpreted as holomorphic mappings between analytic spaces. Somewhat surprisingly, it is often possible to go the other way, to interpret analytic objects in an algebraic way.

       Actually,(Reference: The Red Book \S 10)this is because the strong topology in $X$ induces a strong structure sheaf $\Omega_x(U)$, which is called the "holomorphic function" on $U$, such that:
       \begin{enumerate}
       \item $O_x(U) \subset \Omega_x(u)$, i.e., All regular function is holomorphic.
       \item All morphisms $f:X\rightarrow Y$ are "holomorphic", i.e., $g\in\Gamma(\Omega_y)$, $f^*g\in\Gamma(\Omega_X)$.
       \item $\Omega_Z)$ on locally closed subvariety $Z$ is induced by the restriction on $\Omega_x(U)$.
       \end{enumerate}

       \noindent$\underline{\bold{Ex}}$ Riemann sphere. \qquad it is easy to prove that the analytic functions from the Riemann sphere to itself are either the rational functions or the identically infinity function (an extension of Liouville's theorem). This fact shows that there is no essential difference between the view of $\hat C$ or $\mathbb{P}^1$

       Here are some important result:
       \begin{thm}Compact riemann surfaces are projective variety.
       \end{thm}

       \noindent$\underline{\bold{Ex}}$ The torus $\mathbb{C}/\Lambda$. Its function field (meromorphic function) over $\mathbb{C}$ are generated by Weierstrass function $\wp(z)$ and $\wp\prime(z)$. (Reference: GTM228 \S 7.5)  There is a equation
         \[\wp '(z)^{2}=4\wp (z)^{3}-g_{2}\wp (z)-g_{3},\] Therefore the function field is isomorphic to $Q(\mathbb{C}[x,y,z]/_{y^2z-4x^2z+g_2xz^2+g_3z^3})$ which determines a elliptic curve.

       \begin{thm}[Chow] Any closed analytic subspace of complex projective space is an algebraic subvariety.\end{thm}

       \begin{thm}[Lefschetz principle] $(m\models ACF-_0) \Rightarrow (m\models Th((\mathbb{C},0,1,+,\times)))$ \end{thm}

       \begin{thm}Statement of GAGA\end{thm}
       \end{document}
